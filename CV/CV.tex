%%%%%%%%%%%%%%%%%%%%%%%%%%%%%%%%%%%%%%%%%
% This document is based on a template available at
% http://www.LaTeXTemplates.com
%
% Original author:
% Trey Hunner (http://www.treyhunner.com/)
%%%%%%%%%%%%%%%%%%%%%%%%%%%%%%%%%%%%%%%%%

%----------------------------------------------------------------------------------------
%	PACKAGES AND OTHER DOCUMENT CONFIGURATIONS
%----------------------------------------------------------------------------------------

\documentclass{resume} % Use the custom resume.cls style

\usepackage[left=0.75in,top=0.6in,right=0.75in,bottom=0.6in]{geometry} % Document margins
\usepackage{fancyhdr}
\usepackage{lastpage}
\usepackage{amssymb}
\usepackage{url}

%\input{officialheader.tex} %If you want this file to compile to my CV, comment out this line and uncomment the next three
\name{Jesús N. Pinto Ledezma} 
\address{\url{jpintole@umn.edu} \\ \url{www.jesusnpl.github.io}  \\ @JesusNPL } 
\address{Department of Ecology, Evolution and Behavior \\ University of Minnesota \\ St Paul, MN 55108} 

\pagestyle{fancy}
\fancyhf{}
\renewcommand{\headrulewidth}{0pt}
\cfoot{Pinto-Ledezma, page \thepage\ of \pageref{LastPage}}
\begin{document}

\begin{rSection}{Research Interests}
I am an evolutionary and quantitative ecologist whose work focuses on developing a deeper understanding of species coexistence and patterns of diversity across spatial and temporal scales, and the underlying processes that drive, maintain and alter such patterns. I have a passion for science and for diversity and inclusion in education and research.
\end{rSection}

%----------------------------------------------------------------------------------------
%	EDUCATION
%----------------------------------------------------------------------------------------

\begin{rSection}{Education}
{\bf Ph.D., Ecology and Evolution} \hfill 2013-2017 \\ 
{\bf Universidade Federal de Goiás - Goiânia} \hfill Goiás, Brazil \\
{Dissertation: \emph{Origin and assembly of Furnariides assemblages across space and time: the role of historical processes}} \\
{Advisor: José Alexandre Felizola Diniz-Filho} \smallskip 

{\bf M.S., Wildlife Management} \hfill 2006-2009 \\
{\bf Universidad Nacional de Córdoba - Córdoba} \hfill Córdoba, Argentina \\
{Dissertation: \emph{Determination of special protected areas for the conservation of migratory birds in the Mar Chiquita Reserve}} \\
{Advisors: Adrian H. Farmer and Enrique H. Bucher}\smallskip 

{\bf B.A., Biology} \hfill 2001-2006 \\
{\bf Universidad Autónoma Gabriel René Moreno} \hfill Santa Cruz, Bolivia \\
{Distinction in All Subjects. \emph{Cum Laude Honors}} \\ 
{Advisor: Teresa Ruiz de Centurión}\smallskip 
\end{rSection}

%----------------------------------------------------------------------------------------
%	ACADEMIC APPOINTMENTS
%----------------------------------------------------------------------------------------


\begin{rSection}{Professional Appointments}
{\bf Research Scientist} \hfill {2020-present} \\
{University of Minnesota, Department of Ecology Evolution and Behavior}\hfill {St Paul, MN, USA} 

{\bf Grand Challenge in Biology Postdoctoral Fellow} \hfill {2017-2020} \\
{University of Minnesota, Department of Ecology Evolution and Behavior}\hfill {St Paul, MN, USA} 

{\bf Research Associate} \hfill {2009-present} \\
{Museo de Historia Natural Noel Kempff Mercado}\hfill {Santa Cruz, Bolivia} \\
\emph{Ad Honorem} 

{\bf Guest Lecturer} \hfill {2012-2013} \\
{Carrera de Biología, Universidad Autónoma Gabriel René Moreno}\hfill {Santa Cruz, Bolivia} 

{\bf Visiting Researcher} \hfill {2010-2011} \\
{Centro de Pesquisas do Pantanal, Universidade Federal de Mato Grosso}\hfill {Cuiabá, Bolivia} 

{\bf Intern} \hfill {2003-2006} \\
{Museo de Historia Natural Noel Kempff Mercado}\hfill {Santa Cruz, Brazil} 

{\bf Bolivian Military Service} \hfill {2000-2001} \\
{Air Force}\hfill {Santa Cruz, Bolivia} 
\end{rSection}

\clearpage

%----------------------------------------------------------------------------------------
%	AWARDS AND FELLOWSHIPS
%----------------------------------------------------------------------------------------

\begin{rSection}{Awards and Fellowships}
\begin{esSubsection}{AAAS/Science Membership Award,}{ the American Association for the Advancement of Science Program for Excellence in Science}{2020-present}{}{}
\end{esSubsection}
\begin{esSubsection}{Grand Challenges in Biology Postdoctoral Program,}{ University of Minnesota, College of Biological Sciences }{2017-2020}{}{}
\end{esSubsection}
\begin{esSubsection}{CAPES PhD fellowship,}{ Coordination for the Improvement of Higher Education Personnel, Brazil }{2015-2017}{}{}
\end{esSubsection}
\begin{esSubsection}{OEA-CGUB Doctoral Scholarship,}{ Organization of American States (OAS) and the Coimbra Group of Brazilian Universities (GCUB), Brazil }{2014-2015}{}{}
\end{esSubsection}
\begin{esSubsection}{Master’s Program in Wildlife Management,}{ US Wildlife Service, Universidad Nacional de Córdoba, Córdoba, Argetina }{2006-2008}{}{}
\end{esSubsection}
\begin{esSubsection}{ISSLR Membership and Travel Award,}{ International Society of Salt Lake Research }{2011}{}{}
\end{esSubsection}
\begin{esSubsection}{SWS Membership Award,}{ Society of Wetlands Scientists }{2010-2013}{}{}
\end{esSubsection}
\begin{esSubsection}{SCB Membership Award,}{ Society for Conservation Biology - A global community of conservation professionals }{2007-2009}{}{}
\end{esSubsection}
\begin{esSubsection}{Best Student Award for the Biology Major,}{ Universidad Autónoma Gabriel René Moreno, Santa Cruz de la Sierra, Bolivia}{2005}{}{}
\end{esSubsection}{}
\end{rSection}


%----------------------------------------------------------------------------------------
%	RESEARCH ACTIVITIES
%----------------------------------------------------------------------------------------

\begin{rSection}{Research Activities}
%\textbf{PENDING FEDERAL GRANTS:}

\textbf{FEDERALLY FUNDED GRANTS }(3 awards totaling {\bf USD 5,886,664})

\begin{pSubsection}{NASA ROSES Biodiversity: }{Mapping changes in forest diversity and disease in North American temperate forests.}{2021-2024}{Role: Co-Investigator}{Cavender-Bares, Jeannine (Lead-PI, UMN), Townsend, Philip (co-PI, UW). {\bf Award: USD 481,933.}}
\end{pSubsection}

\begin{pSubsection}{National Science Foundation, MSA: }{Integrating biodiversity observations with airborne and satellite data to predict shifts in assemblage diversity and composition under global change.}{2020-2023}{Role: Principal Investigator}{}{{\bf Pinto-Ledezma, Jesús N.} (Lead-PI, UMN), Cavender-Bares, Jeannine (co-PI, UMN). {\bf Award: USD 299,375.}}
\end{pSubsection}

\begin{pSubsection}{National Science Foundation, BII Implementation: }{The causes and consequences of plant biodiversity across scales in a rapidly changing world.}{2020-2025}{Role: Co-Investigator}{Cavender-Bares, Jeannine (Lead-PI, UMN), Townsend, Philip (co-PI, UW), Reich, Peter (co-PI, UMN), José E. Meireles (co-PI, UMaine), Amy Trowbridge (co-PI, UW). More information at: \url{https://www.spectralbiology.org}. {\bf Total award: USD 12,5000,000.} Awarded to date: USD 5,105,356.}
\end{pSubsection}

\textbf{NON-FEDERALLY FUNDED GRANTS }(8 awards totaling {\bf USD 188,508})

\begin{pSubsection}{College of Biological Sciences, UMN, Grand Challenges in Biology Postdoctoral Fellowship: }{Evaluating the roles of ecological and historical processes in biological invasions.}{2017-2020}{Role: Postdoctoral Fellow}{{\bf Award: USD 157,500.}}
\end{pSubsection}

\begin{pSubsection}{Academia Nacional de Ciencias de Bolivia, Capitulo Santa Cruz: }{Amphibians as a model of biological control in agricultural areas of central Santa Cruz, Bolivia.}{2016-2017}{Role: Co-Principal Investigator}{Pinto, Marco Aurelio (Lead-PI), \textbf{Pinto-Ledezma, Jesús N.} (Co-PI). {\bf Award: USD 1,500.}}
\end{pSubsection}

\begin{pSubsection}{Rufford Foundation: }{Rescuing the biodiversity of the Cerro Mutún: a basis for generation the conservation measures for Bolivian biodiversity.}{2016-2017}{Role: Co-Principal Investigator}{Villarroel, Daniel (Lead-PI), \textbf{Pinto-Ledezma, Jesus N.} (Co-PI). {\bf Award: USD 7,674.}}
\end{pSubsection}

\begin{pSubsection}{Rufford Foundation: }{Long-Term Effects of Habitat Modification on Amphibians in the Yungas and Inter-Andean Dry Valley Ecoregions.}{2013-2014}{Role: Co-Principal Investigator}{Sosa, Ronald (Lead-PI), \textbf{Pinto-Ledezma, Jesús N.} (Co-PI). {\bf Award: USD 6,568.}}
\end{pSubsection}

\begin{pSubsection}{Rufford Foundation: }{The Hyacinth Macaw Program: Population Status and Conservation of the Hyacinth Macaw.}{2013-2014}{Role: Principal Investigator}{\textbf{Pinto-Ledezma, Jesús N.} (PI). {\bf Award: USD 7,168.}}
\end{pSubsection}

\begin{pSubsection}{Academia Nacional de Ciencias de Bolivia, Capítulo Santa Cruz: }{Analysis of effect of the land use change on amphibian communities in the Mutun region.}{2012-2013}{Role: Principal Investigator}{\textbf{Pinto-Ledezma, Jesús N.} (PI). {\bf Award: USD 1,500.}}
\end{pSubsection}

\begin{pSubsection}{Academia Nacional de Ciencias de Bolivia, Capítulo Santa Cruz: }{Areas for the conservation of the Hyacinth macaw.}{2011-2012}{Role: Principal Investigator}{\textbf{Pinto-Ledezma, Jesús N.} (PI). {\bf Award: USD 1,500.}}
\end{pSubsection}

\begin{pSubsection}{Rufford Foundation: }{Testing a Habitat Model for the Hyacinth macaw (\emph{Anodorhynchus hyacinthinus}) and Mapping HS for the Species in Protected Areas in Bolivian Pantanal.}{2009-2011}{Role: Principal Investigator}{\textbf{Pinto-Ledezma, Jesús N.} (PI). {\bf Award: USD 5,098.}}
\end{pSubsection}

\end{rSection}


%----------------------------------------------------------------------------------------
%	PUBLICATIONS
%----------------------------------------------------------------------------------------

\begin{rSection}{Publications}
As of May 2021, I have published 30 peer-reviewed articles, 4 peer-reviewed book chapters and 3 non-peer-reviewed chapters.

\textbf{Indexed Journals and Peer-reviewed Book chapters:}

\em{*Undergraduate student}

{\em 34.} {\em Cavender-Bares, J., P. Reich, P.A. Townsend, A. Banerjee, E. Butler, A. Desai, A. Gevens, S. Hobbie, F. Isbell, E. Laliberté, J.E. Meireles, H. Menninger, R.P. Pavlick, {\bf{J.N. Pinto-Ledezma}}, C. Potter, M.C. Schuman, N. Springer, A. Stefanski, P. Trivedi, A. Trowbridge, L. Williams, C.G. Willis and Y. Yang. (2021). BII-Implementation: The causes and consequences of plant biodiversity across scales in a rapidly changing world.} {Research Ideas and Outcomes, 7: e63850}. 

{\em 33.} {\bf{Pinto-Ledezma, J.N.}}, {\em F. Villalobos, P. Reich, J. Catford, D. Larkin and J. Cavender-Bares. (2020). Testing Darwin's naturalization conundrum based on taxonomic, phylogenetic and functional dimensions of vascular plant diversity.} {Ecological Monographs, 90(4): e01420}. 

{\em 32.} {\em Cavender-Bares, J., C. Fontes and} {\bf{J.N. Pinto-Ledezma}}. {\em (2020). Open questions in understanding the adaptive significance of plant functional trait variation within a single lineage.} {New Phytologist, 227(3): 659-663}. 

{\em 31.} {\bf{Pinto-Ledezma, J.N.}} {\em and J. Cavender-Bares. (2020). Using remote sensing for modeling and monitoring species distributions. In Cavender-Bares, J., J. Gamon and P. Townsend (Eds.)} {Remote Sensing of Plant Biodiversity. Springer Remote Sensing/Photogrammetry Series}.

{\em 30.} {\em Cavender-Bares, J., A. Schweiger,} {\bf{J.N. Pinto-Ledezma}} {\em and J.E. Meireles. (2020). Applying remote sensing to biodiversity science. In Cavender-Bares, J., J. Gamon and P. Townsend (Eds.)} {Remote Sensing of Plant Biodiversity. Springer Remote Sensing/Photogrammetry Series}. 

{\em 29.} {\em Villalobos, F.,} {\bf{J.N. Pinto-Ledezma}} {\em and J.A.F. Diniz-Filho. (2020). Evolutionary macroecology and the geographical patterns of Neotropical diversification. In Rull, V. and A.C. Carnaval (Eds.)} {Neotropical diversification: patterns and processes. Springer Nature AG}. 

{\em 28.} {\bf{Pinto-Ledezma, J.N.}}, {\em A.E. Jahn, V.R. Cueto, J.A.F. Diniz-Filho and F. Villalobos. (2019). Drives of phylogenetic assemblage structure of the Furnariides, a widespread clade of lowland Neotropical birds.} {The American Naturalist, 193(2): E41-E5}. 

{\em 27.} {\bf{Pinto-Ledezma, J.N.}}, {\em D. Larkin and J. Cavender-Bares. (2018). Patterns of beta diversity of vascular plants and their correspondence with biome boundaries across North America.} {Frontiers in Ecology and Evolution, 6: 194}.

{\em 26.} {\em Contributing author in: Cavender-Bares, J. et al. Chapter 3 Status and trends of biodiversity and ecosystem functions underpinning nature’s benefit to people. In IPBES (2018): {\em The IPBES regional assessment report on biodiversity and ecosystem services for the Americas}. 207-362 Pp. Rice et al. (Eds).} {Secretariat of the Intergovernmental Science-Policy Platform on Biodiversity and Ecosystem Services, Bonn, Germany}. 

{\em 25.} {\em Pereira, E.,} {\bf{J.N. Pinto-Ledezma}}, {\em C. de Freitas, F. Villalobos, R. Collevati and N. Medeiros. (2017). Evolution of anuran foam nest: trait conservatism and lineage diversification.} {Biological Journal of the Linnean Society 122(4): 814-823}. 

{\em 24.} {\bf{Pinto-Ledezma, J.N.}}, {\em L. Simon, J.A.F Diniz-Filho and F. Villalobos. (2017) The geographic diversification of Furnariides: the role of forest versus open habitats in driving species richness gradients.} {Journal of Biogeography, 44(8): 1683-1693}. 

{\em 23.} {\bf{Pinto-Ledezma, J.N.}}, {\em M.A. Montenegro and D. Villarroel. (2017). Historia Natural del Cerro Mutún V: la avifauna.} {Kempffiana, 13(2): 10-28}.

{\em 22.} {\em Cseko, E., W. Franca-Rocha, T. Moura and} {\bf{J.N. Pinto-Ledezma}}. {\em (2017). New range limit of the Broad-tipped Hermit ({\em Anopetia gounellei}, Aves: Trochilidae): the state of art and a review on the range area.} {Pápeis avulsos de Zoologia, 57(21): 275-285}. 

{\em 21.} {\em Villarroel, D., G. Aramayo, M. Martínez, C. Proença, C. Munhoz, B. Klitgaard,} {\bf{J.N. Pinto-Ledezma}} {\em and M. Nee. (2017) Historia Natural del Cerro Mutún VI: flora y vegetación, checklist, estado de conservación y nuevos registros para Bolivia.} {Kempffiana, 13(2): 29-74}.

{\em 20.} {\em *Pinto, M.A., *K. Mano-Cuellar, D. Villarroel and} {\bf{J.N. Pinto-Ledezma}}. {\em (2017). Historia Natural del Cerro Mutún IV: la herpetofauna.} {Kempffiana, 13(1): 116-128}.

{\em 19.} {\bf{Pinto-Ledezma, J.N.}} {\em and D. Villarroel. (2016). Historia Natural del Cerro Mutún I: síntesis geográfica, geofísica, climática y socioeconómica.} {Kempffiana, 12(2): 29-38}.

{\em 18.} {\em *Pinto, M.A. and} {\bf{J.N. Pinto-Ledezma}}. {\em (2015). Listado preliminar de anfibios de la propiedad Benevento (Santa Cruz, Bolivia).} {Kempffiana, 11(1): 23-27}.

{\em 17.} {\em *Pinto M.A., D. García, K. Mano and} {\bf{J.N. Pinto-Ledezma}}. {\em (2015). Listado de anfibios y reptiles de la propiedad Juan Deriba, Santa Cruz, Bolivia.} {Kempffiana, 11(1): 70-75}.

{\em 16.} {\em *Mano K., *M.A. Pinto, *R. Sosa, D. Villarroel and} {\bf{J.N. Pinto-Ledezma}}. {\em (2015). Reptile fauna of the Mutún region (Santa Cruz department, Bolivia): species list and conservation status.} {Kempffiana, 11(1): 66-69}.

{\em 15.} {\em *Sosa R., Ch. Schalk, L. Braga and} {\bf{J.N. Pinto-Ledezma}}. {\em (2015). Geographic Distribution: {\em Rhinella amboroensis} (Cochabamba toad)} {Herpetological Review, 46(2): 214}.

{\em 14.} {\bf{Pinto-Ledezma, J.N.}} {\em and *M.L. Rivero. (2014) Temporal patterns of deforestation and fragmentation in Lowland Bolivia: Implications for climate change.} {Climatic Change, 127: 43-54}. 

{\em 13.} {\bf{Pinto-Ledezma, J.N.}}, {\em T.J. Caballero, B. Flores, V.N. Perez, *K. Mano and *M.A. Pinto. (2014). Lista preliminar de las aves de la propiedad Juan Deriba, Santa Cruz, Bolivia.} {Kempffiana, 10(2): 1-11}.

{\em 12.} {\em *Sosa R., L. Braga and} {\bf{J.N. Pinto-Ledezma}}. {\em (2014). The amphibian fauna of the Southwest Amboró National Park, Santa Cruz, Bolivia.} {Kempffiana, 10(2): 31-35}.

{\em 11.} {\bf{Pinto-Ledezma, J.N.}}, {\em V.X. Sandoval, V.N. Pérez, T.J. Caballero, *K. Mano, *M.A. Pinto and *R. Sosa. (2014). Desarrollo de un modelo espacial explícito de hábitat para la paraba jacinta ({\em Anodorhynchus hyacinthinus}) en el Pantanal boliviano (Santa Cruz, Bolivia). } {Ecología en Bolivia, 49(2): 1605-2528}.

{\em 10.} {\em Jahn A.E., D.J. Levey, V. Cueto,} {\bf{J.N. Pinto-Ledezma}}, {\em D. Tuero, J.W. Fox and D. Masson. (2013). Patterns of long-distance bird migration in South America as revealed by light-level geolocators.} {The Auk, 130(2): 223-229}. 

{\em 9.} {\em Jahn A.E., V. Cueto, J.W. Fox, M.S. Husak,} {\bf{J.N. Pinto-Ledezma}}, {\em D.H. Kim, D.V. Landoll, H.K. Lepage, D.J. Levey, M.T. Murphy and R.B. Renfrew. (2013) Migration timing and wintering areas of three species of Tyrannus flycatchers breeding in the great plains of North America} {The Auk, 130(2): 247-257}. 

{\em 8.} {\bf{Pinto-Ledezma, J.N.}} {\em and M.A. Aponte. (2013) Algunas notas sobre la reproducción de aves en la Reserva de Vida Silvestre Ríos Blanco y Negro.} {Kempffiana, 9(1): 21-25}.

{\em 7.} {\em *Sosa R., Ch. Schalk, L. Braga and} {\bf{J.N. Pinto-Ledezma}}. {\em (2013). {\em Micrurus serranus} (NCN) diet.} {Herpetological Review, 44(1): 155}. 

{\em 6.} {\em *Sosa R., Ch. Schalk, L. Braga and} {\bf{J.N. Pinto-Ledezma}}. {\em (2013). {\em Phylodryas psammohidea} (Gunther's green racer) diet.} {Herpetological Bulletin, 124: 24}. 

{\em 5.} {\em *Sosa R., Ch. Schalk, L. Braga and} {\bf{J.N. Pinto-Ledezma}}. {\em (2012). {\em Clelia langeri} (NCN) diet.} {Herpetological Review, 43(4): 657}. 

{\em 4.} {\bf{Pinto-Ledezma, J.N.}}, {\em aR. Sosa, M. Paredes, I. García, D. Villarroel and S. Muyucundo. (2011). The Hyacinth macaw ({\em Anodorhynchus hyacinthinus}): population status and its conservation in Bolivian Pantanal.} {Kempffiana, 7(2): 15-37.}. 

{\em 3.} {\em Jahn A.E.,} {\bf{J.N. Pinto-Ledezma}}, {\em A.M. Mamani, L.W. De Groote and D.J. Levey. (2010). Patterns of home range size and habitat occupancy of Tropical Kingbird ({em\ Tyrannus m. melancholicus}) in the southern Amazon Basin.} {Ornitología Neotropical, 12: 39-46}. 

{\em 2.} {\bf{Pinto-Ledezma, J.N.}} {\em and T. Ruiz de Centurión. (2010). Patrones de deforeatación y fragmentación en el Municipio de San Julián, periodo 1976 y 2006.} {Ecología en Bolivia, 45(2): 101-115}.

{\em 1.} {\em Villarroel D.,} {\bf{J.N. Pinto-Ledezma}}, {\em T. Ruiz de Centurión and A. Parada. (2009) Relaciones entre la cobertura arbórea y herbácea en tres fisonomías del Cerrado {\em sensu lato} (Cerro Mutún, Santa Cruz, Bolivia).} {Ecología en Bolivia, 44(2): 83-98}. 

{\bf Non-per-reviewed book chapters:}

{\em 3.} {\em Mostacedo B., M., Toledo, D. Villarroel,} {\bf{J.N. Pinto-Ledezma}}, {\em G. Carreño-Rocabado, B. Flores and Y. Uslar. (2014). Memorias del IV Congreso Boliviano de Ecología. 4-6 de Junio 2014. Universidad Autónoma Gabriel Rene Moreno, Santa Cruz, Bolivia.}

{\em 2.} {\em Azurduy H. and} {\bf{J.N. Pinto-Ledezma}} {\em (2012). El escenario ecológico y geográfico. 6-13 pp. In: Azurduy and Rivero (Eds). Historia Natural de la Serranía Incahuasi. NHNNKM-Total.} {Museo de Historia Natural Noel Kempff Mercado}. 

{\em 1.} {\em Villarroel D., L. Arroyo and} {\bf{J.N. Pinto-Ledezma}} {\em (2012). (2009). La vegetación de Bella Vista. 45-68 Pp. In: Arroyo and Churchill (Eds). Investigaciones botánicas en la región de Bella Vista, departamento de Santa Cruz, Bolivia: una base para la conservación.} {Museo de Historia Natural Noel Kempff and Missouri Botanical Garden}. 

\end{rSection}

%----------------------------------------------------------------------------------------
%	OTHER WRITING
%----------------------------------------------------------------------------------------

\begin{rSection}{Intellectual Contributions in Review or Submitted:}

{6.} {Fontes, C.,} {\bf{J.N. Pinto-Ledezma}}, {A.L. Jacobsen, R.B. Pratt and J. Cavender-Bares. (First revision). Adaptive variation among oaks in wood anatomical properties is shaped by climate of origin and shows limited plasticity across environments.} {\em New Phytologist.}

{5.} {\bf{Pinto-Ledezma, J.N.}} {and J. Cavender-Bares. (First revision). Predicting species distributions and community composition using satellite remote sensing predictors.} {\em Scientific Reports.} {Available online at:} \url {https://doi.org/10.21203/rs.3.rs-345639/v1}.

{4.} {Mendes, L.,} {\bf{J.N. Pinto-Ledezma}}, {T.F.L.V.B. Rangel and R. Dunn. (First revision). Urban warming inverse contribution on risk of dengue transmission in the southeastern North America.} {\em Proceedings of the Royal Society: Biological Sciences.} {Available online at:} \url {https://www.biorxiv.org/content/10.1101/2020.01.15.908020v1}.

{3.} {Velasco, J.A., G. Campillo-García,} {\bf{J.N. Pinto-Ledezma}}, {and O. Villela-Flores. (Second revision) Spatiotemporal dimensions of a reproductive life history trait in a spiny lizard radiation (Squamata: Phrynosomatidae).} {\em Proceedings of the Royal Society: Biological Sciences.} {Available online at:} \url {https://www.biorxiv.org/content/10.1101/2020.06.17.157891v1}.

{2.} {Velasco, J.A. and} {\bf{J.N. Pinto-Ledezma}}. {(First revision). Mapping diversification metrics in macroecological studies: prospects and challenges.} {\em Ecography.} {Available online at:} \url {https://www.biorxiv.org/content/early/2018/02/08/261867.1}.

{1.} {Souza, K.,} {\bf{J.N. Pinto-Ledezma}}, {R. Dobrovolski, M. Telles, T. Soares, C. Ruas and J.A.F. Diniz-Filho. (First revision). How to measure the influence of landscape population genetic structure: developing resistance surfaces using a pattern-oriented modeling approach.} {\em Genetica.} {Available online at:} \url {https://www.biorxiv.org/content/10.1101/2020.02.20.958637v1?rss=1}.

\end{rSection}

%----------------------------------------------------------------------------------------
%	WRITING IN PROGRESS
%----------------------------------------------------------------------------------------

\begin{rSection}{Intellectual Contributions in Preparation:}

{4.} {\bf{Pinto-Ledezma, J.N.}}, {L. Kuczynski, J.A. Velasco, K. Marske, A. Carnaval, M. Papes, J. Cavender-Bares. Trait biogeography: the legacies of evolution and biogeographical origins.} {\em Target Journal: PNAS}.  

{3.} {\bf{Pinto-Ledezma, J.N.}}, {J. Cavender-Bares +NutNet group. Evolutionary legacies on ecosystems: detecting phylogenetic responses of plants to global change.} {\em Target Journal: Science Advances}.  

{2.} {\bf{Pinto-Ledezma, J.N.}}, {J.E. Meireles, F. Villalobos. Splendid isolation: diversification dynamics of the largest continental endemic vertebrate radiation.} {\em Target Journal: Evolution}. 

{1.} {Rueda-Cediel, P., R. Brain, N. Galic, {\bf J.N. Pinto-Ledezma}, A. Rico and V. Forbes. Characterization of sensitivity patterns across life-history groupings of US herbaceous plants to inform pesticide risk assessments.} {\em Target Journal: Science of The Total Environment}. 
\end{rSection}


%----------------------------------------------------------------------------------------
%	TEACHING
%----------------------------------------------------------------------------------------

\begin{rSection}{Teaching and Advising}
\textbf{CORE TEACHING:}

\begin{reSubsection}{University of Minnesota: }{Department of Ecology, Evolution \& Evolution }{}{
EEB 3534: \textbf{Biodiversity Science} (Co-taught with Prof. Jeannine Cavender-Bares) \hfill Spring 2021 \\ 
EEB 5534: \textbf{Biodiversity Science} (Co-taught with Prof. Jeannine Cavender-Bares) \hfill Spring 2021 \\ 
{Lab material at:
\url{https://jesusnpl.github.io/teaching/}}\smallskip

EEB 3534: \textbf{Biodiversity Science} (Main Instructor) \hfill Spring 2020 \\ 
EEB 5534: \textbf{Biodiversity Science} (Main Instructor) \hfill Spring 2020 \\ 
{Lab material at:
\url{https://github.com/jesusNPL/BiodiversityScience/Spring2020}}\smallskip 

EEB 5534: \textbf{Biodiversity Science} (Co-taught with Prof. Jeannine Cavender-Bares) \hfill Spring 2019 \\  
EEB 3534: \textbf{Biodiversity Science} (Co-taught with Prof. Jeannine Cavender-Bares) \hfill Spring 2019 \\ 
{Lab material at:
\url{https://github.com/jesusNPL/BiodiversityScience/Spring2019}}\smallskip 
}
\end{reSubsection}

\begin{reSubsection}{Universidade Federal de Goiás: }{Department of Ecology }{}{
\textbf{Phylogenetic Comparative Methods} (Teaching Assistant) \hfill Spring 2016 \\ 
{Lab material at:
\url{http://dinizfilho.wixsite.com/dinizfilholab/}}\smallskip 
} 
\end{reSubsection}

\begin{reSubsection}{Universidad Autónoma Gabriel René Moreno: }{Carrera de Biología }{}{
ZOO 344: \textbf{Vertebrate Zoology} (Guest Lecturer) \hfill Spring 2012, 2013 \\ \textbf{Landscape Ecology} (Guest Lecturer) \hfill Spring 2015, 2017 \\
{Master en Manejo de Recursos Naturales y Medio Ambiente} \smallskip 
} 
\end{reSubsection}

\textbf{ADDITIONAL TEACHING:}

{\bf University of Minnesota: }{Department of Fisheries, Wildlie and Conservation Biology } \\
\textbf{Introduction to patterns of biodiversity} (Guest Lecturer) \hfill Oct, 2019 \\
{Lab material at:
\url{https://github.com/jesusNPL/LargeScale}}\smallskip 

{\bf Universidad Autónoma Gabriel René Moreno: }{Carrera de Biología } \\
ZOO 344: \textbf{Vertebrate Zoology} (Teaching Assitant) \hfill Spring 2003-2005, Fall 2003-2005 \\
{Six semesters}\smallskip 

{\bf Universidad Autónoma Gabriel René Moreno: }{Department of Botany } \\
\textbf{Introduction of statistics} (Instructor) \hfill Mar, 2012, 2013, 2014 \\
{Three intensive courses of one week each}\smallskip 

{\bf Universidad Autónoma Gabriel René Moreno: }{IV Congreso Boliviano de Ecología } \\
\textbf{Species distribution modeling with R} (Instructor) \hfill Jun, 2014 \\
{Three days course}\smallskip 

\textbf{DIRECTED STUDENT MENTORING:}

\textbf{PhD Thesis Committee:}

\textbf{Axel Arango García}, August 2019 - Present. External committee member, PhD Thesis/Project: 'Effects of dispersal on the diversification of Emberizoidea (Aves, Passeriformes) in the New World'. Instituto de Ecología A.C., Xalapa, Mexico. 

\textbf{Master's Graduate Advisees:}

\textbf{Marco Aurelio Pinto Viveros}, February 2017 – August 2019. Master Science Thesis: 'The amphibians as a model of biological control in agricultural areas of Santa Cruz, Bolivia', Master program Manejo de Recursos Naturales y Medio Ambiente, Universidad Autónoma Gabriel René Moreno, Santa Cruz de la Sierra, Bolivia.

\textbf{Past Undergrad Advisees:}

\textbf{Katherine Mano Cuellar}, March 2012 – July 2014. Distinction in All Subjects. {\em Magna Cum Laude}. Undergraduate Project. 'Effects of land use change on amphibian community composition in central Santa Cruz, Bolivia'. Carrera de Biología, Universidad Autónoma Gabriel René Moreno, Santa Cruz de la Sierra, Bolivia.

\textbf{Marco Aurelio Pinto Viveros}, March 2012 – December 2014. Distinction in All Subjects. {\em Magna Cum Laude}. Undergraduate Project: 'The herpetofauna of the Mutún region, Santa Cruz, Bolivia'. Carrera de Ciencias Ambientales, Universidad Autónoma Gabriel René Moreno, Santa Cruz de la Sierra, Bolivia.

\textbf{Ronald Sosa Escalante}, March 2013 – July 2016. Undergraduate Thesis: 'Estudio de la mortalidad de serpientes atropelladas en la carretera Antigua a Cochabamba, Provincia Florida, Santa Cruz, Bolivia, Carrera de Biología, Universidad Autónoma Gabriel René Moreno, Santa Cruz de la Sierra, Bolivia.

\end{rSection}


%----------------------------------------------------------------------------------------
%	INVITED TALKS
%----------------------------------------------------------------------------------------

\begin{rSection}{Presentations}

\begin{sSubsection}{Invited talk - Plant diversity: community structure, composition and detection}{ }{Jan, 2021}{ Ecology, Evolution and Behavior Seminar Series, University of Minnesota (Online)}{St Paul, MN}
\end{sSubsection}

\begin{sSubsection}{Poster - Integrating Biodiversity Observations With Airborne and Satellite Data To Predict Shifts in Assemblage Diversity and Composition Underg Global Change}{ }{Jan, 2021}{NSF Macrosystems Biology and NEON Enabled Science PI Meeting}{Online meeting}
\end{sSubsection}

\begin{sSubsection}{Invited talk - Macroecology and macroevolution in the Neotropics}{ }{Nov, 2020}{Department of Geography, Federal University of Rio Grande do Norte}{Online talk}
\end{sSubsection}

\begin{sSubsection}{Talk - Introduction to graphical models}{ }{Jan, 2020}{Evoutionary Biology Network, Institute of Ecology}{Xalapa, Mexico}
\end{sSubsection}

\begin{sSubsection}{Talk - The role of ecology and evolution on the assembly and species co-occurrence at different spatial and temporal scales}{ }{Sep, 2019}{Grand Challenges in Biology Symposium, University of Minnesota}{St Paul, MN}
\end{sSubsection}

\begin{sSubsection}{Poster - Testing Darwin's naturalization conundrum based on taxonomic, phylogenetic and functional dimensions of vascular plant diversity}{ }{Aug, 2019}{Ecological Society of America Annual Meeting}{Louisville, KY}
\end{sSubsection}

\begin{sSubsection}{Invited talk - Wildlife management and indigenous people in Bolivian lowlands}{ }{Apr, 2019}{Mano a Mano International Partners}{St Paul, MN}
\end{sSubsection}

\begin{sSubsection}{Talk - Integrated Global Biodiversity Detection: Plant Spectra, Phylogenetics, and Enhanced Species Distribution Models}{ (Co-author) }{Dec, 2018}{AGU Annual Meeting}{Washington DC}
\end{sSubsection}

\begin{sSubsection}{Talk - Evolutionary macroecology}{ }{Dec, 2018}{Museo de Historia Natural Noel Kempff Mercado}{Santa Cruz, Bolivia}
\end{sSubsection}


\end{rSection}

%----------------------------------------------------------------------------------------
%	ORGANIZATION
%----------------------------------------------------------------------------------------
\begin{rSection}{Symposia and Workshops Organized}

\begin{sSubsection}{Round table on Diversity in Biodiversity Science}{ (Organization committee) }{Oct, 2020}{Biodiversity Research Coordination Network (RCN)}{Online meeting}
\end{sSubsection}

\begin{sSubsection}{X Bolivian Congress of Ornithology }{ (Scientific committee) }{Oct, 2019}{Asociación Boliviana de Ornitología and Universidad }{ Sucre, Bolivia}
\end{sSubsection}

\begin{sSubsection}{IV Bolivian Congress of Ecology }{ (Vice-president and Scientific committee) }{Jun, 2014}{Asociación Boliviana de Ecología }{ Santa Cruz, Bolivia}
\end{sSubsection}

\begin{sSubsection}{Climate Change and Water Use }{ (Organization committee) }{Oct, 2010}{ADAPCLIM conference }{ Asunción, Paraguay}
\end{sSubsection}

\begin{sSubsection}{First Encounter on Knowledge and Management of the Pantanan and Chiquitania in the context of the Paraguay River Basin}{ (Organization committee) }{Jun, 2010}{Museo Noel Kempff Mercado/SINERGIA }{ Puerto Quijarro, Bolivia}
\end{sSubsection}

\end{rSection}

%----------------------------------------------------------------------------------------
%	EXPERIENCE
%----------------------------------------------------------------------------------------
\begin{rSection}{Experience with Scientific Collections and Curation}

\textbf{Museo de Historia Natural Noel Kempff Mercado, Santa Cruz de la Sierra, Bolivia} \smallskip 
\item 1.	\textbf{Assistant curator}: identification, preparation and maintenance of bird specimens collected in the field (2009 – 2014). 
\item 2.	\textbf{Coordinator}: development of the Geospatial Centre for Biodiversity in Bolivia—for the vertebrate collection at the Museum in collaboration with a team of taxonomists. \url{http://www.museonoelkempff.org/cgb/}. (2011 – 2013). 
\item 3.	\textbf{Field coordinator of biological inventories} in National Protected Areas of Bolivia, including Ríos Blanco y Negro Wildlife Reserve (2009), Otuquis National Park (2011, 2013).

\end{rSection}


%----------------------------------------------------------------------------------------
%	SKILLS
%----------------------------------------------------------------------------------------

\begin{rSection}{Skills}

\begin{tabular}{ @{} >{\bfseries}l @{\hspace{6ex}} l }
Languages & First language: Spanish (Fluent speaking, reading, and advanced writing) \\ 
 & Proficient: English, Portuguese (speaking, reading, and writting) \smallskip \\ 

Computer Programming & Advanced: \textsc{R}, \textsc{Markdown} \\
 & Intermediate: \LaTeX, \textsc{Git}, \textsc{ArcGIS}, \textsc{QGIS}, \textsc{ERDAS Imagine} \\ 
 & Familiar with: \textsc{Python}, \textsc{C++}, \textsc{Matlab}, \textsc{Bash}, \textsc{Envi}

\end{tabular}

%Open source contributions: \\
%Tidy labs for Introduction to Statistical Learning: \url{https://github.com/SmithCollege-SDS/tidy-islr} \\ 
%Labs accompanying OpenIntro Introduction to Statistics with Randomization and Simulation \url{https://github.com/beanumber/oiLabs-mosaic} \\ 
%R packages: 
%Many more projects hosted on GitHub, \url{https://github.com/AmeliaMN} \\
%Additional projects contributed to: 
\end{rSection}



%----------------------------------------------------------------------------------------
%	SERVICE
%----------------------------------------------------------------------------------------
\begin{rSection}{Service and Outreach Activities}

\textbf{SERVICE AS REVIEWER:} \smallskip \\ 
\textbf{Publons}: \url{https://publons.com/researcher/1588941/jesus-n-pinto-ledezma/}

I served as an expert reviewer for the 2020 Red List of birds (eastern South America). In addition, I have served as a reviewer in areas of macroecology, macroevolution, bird taxonomy and systematics, ecology and evolution, community ecology in journals, including:

Ecology(3), Frontiers in Ecology and the Environment, Ecological Monographs(2), Ecology Letters, Journal of Biogeography(8), Global Ecology and Biogeography(7), Diversity and Distributions(3), Ecology and Evolution(2), Journal of Field Ornithology, EMU (Australian Journal of Ornithology), El Hornero (Ornitología Neotropical), Kempffiana, Oecología Australis, Journal of Aninal Ecology(2), Journal of Ecology, Journal of Plant Ecology, Annals of Botany, Biodiversity and Conservation(2), Forest Ecology and Management, Plos One, Prespectives in Ecology and Conservation(2), Journal of Zoological Systematics and Evolutionary Research, Nature Communications(2), Systematic Biology, Molecular Biology and Evolution, Biological Journal of the Linnean Society(2), The American Naturalist, Journal of Vegetation Science(2), New Phytologist, Remote Sensing in Ecology and Conservation, Methods in Ecology and Evolution. 


\textbf{NATIONAL AND INTERNATIONAL SERVICE ACTIVITIES:} \smallskip 
\item 1. \textbf{Official Bolivian member} for the Society of Wetlands Scientists, South American Chapter. 
\item 2. \textbf{Research Advisor in Ecology and Natural Resources}: National Academy of Sciences of Bolivia, Santa Cruz Chapter ({\em Scientia Crucensis}). 

\textbf{INTERNSHIP IN NATIONAL PROTECTED AREAS:} \smallskip 
\item 1. \textbf{Parque Nacional Leoncito}, Departamento Calingasta, San Juan, Argentina. 21 days including weekends. (2007).
\item 2. \textbf{Reserva Forestal el Choré}, Provincia Ichilo, Santa Cruz, Bolivia. 45 days including weekends. (2005). 

\textbf{OUTREACH:} \smallskip \\ 
While working at the Noel Kempff Mercado Natural History Museum in Santa Cruz de la Sierra, I participated in outreach activities with visitors. I participated in guided visits from elementary and high school students and to the ornithological collection at the Museum, where we explored the role of scientific collections in science and society and how we learn about and document biodiversity.

As a postdoctoral fellow at the university of Minnesota, I have been participating in different science outreach programs. One of the programs involves bringing elementary school students (usually 5th and 6th graders) to the University of Minnesota. I lead part of the biodiversity sessions, in which the students can see and manipulate different plant species in the greenhouse and learn about the role of environmental conditions in species diversity, function and adaptations. I have also participated in the Market Science, which aims to connect people with science. In Market Science we created hand-one science activities for children in Farmer’s markets. For example, my last experience in Market Science (September 2019), Laura Toro (PhD candidate from Colombia) and I created activities related to the impacts of wildfire and human induced fires in South America. Also, I gave a public talk at “Mano a Mano” International Partners (April 2019, \url{https://manoamano.org}) on wildlife management and indigenous people in Bolivian lowlands. In this talk, I highlighted the importance of working with local communities in order to preserve the natural capital. More recently, I have co-organized an online round table entitled {\bf Diversifying Biodiversity Science} under the RCN: Cross-Scale Processes Impacting Biodiversity collaborative project, in which several panelist were invited to talk about their experiences regarding the inequalities in biodiversity science and how they are working to make our work more inclusive.

I am also a Spanish-language reviewer for {\bf The American Naturalist}. The aim is helping The American Naturalist to expand the communications reach of the world-class science that nonnative English speakers produce.


\end{rSection}

%----------------------------------------------------------------------------------------
%	MEMBERSHIPS
%----------------------------------------------------------------------------------------

\begin{rSection}{Memberships in Professional Societies}
\begin{esSubsection}{American Association for the Advancement of Science }{(AAAS)}{2020-present}{}{}
\end{esSubsection}

\begin{esSubsection}{Ecological Society of America }{(ESA)}{2014-present}{}{}
\end{esSubsection}

\begin{esSubsection}{National Academy of Sciences of Bolivia, Santa Cruz Chapter }{({\em Scientia Crucensis})}{2013-Present}{}{}
\end{esSubsection}


\begin{esSubsection}{Ornithological Society of Bolivia }{(ASBOR)}{2012-Present}{}{}
\end{esSubsection}

\begin{esSubsection}{Ecological Society of Argentina }{(ASAE)}{2006-Present}{}{}
\end{esSubsection}

\begin{esSubsection}{Ecological Society of Bolivia }{(ABECOL)}{2006-Present}{}{}
\end{esSubsection}

\begin{esSubsection}{Society of Wetlands Scientists, South American Chapter }{(SWS)}{2010-2014}{}{}
\end{esSubsection}

\begin{esSubsection}{International Society for Salt Lake Research }{(ISSLR)}{2010-2014}{}{}
\end{esSubsection}

\begin{esSubsection}{Community of Wildlife Management in Latin America }{(COMFAUNA)}{2009-Present}{}{}
\end{esSubsection}

\begin{esSubsection}{Society of Conservation Biology, Bolivian Chapter }{(SCB)}{2007-2012}{}{}
\end{esSubsection}

\end{rSection}



%----------------------------------------------------------------------------------------
%	References
%----------------------------------------------------------------------------------------
\clearpage
%\input{references.tex} %Comment out to get this to compile




\end{document}
