\documentclass[margin,line]{res}

% File produced by Jeremy West
% This file may be distributed and/or modified:
%   1. under the LaTeX Project Public License and/or
%   2. under the GNU Public License.

% NOTE: you also need the (included) res.cls class file to compile this document
% Either place the .cls file in the same directory (folder) as this .tex file, or install it globally for your LaTeX distribution (search online for how to do so).

\usepackage[dvipsnames]{xcolor}
\usepackage[export]{adjustbox}
\usepackage{graphicx}
\usepackage{times}
\usepackage{setspace}                                    % Allows for custom margins, etc.
\usepackage{fullpage}                                    % Use the full page
\usepackage[utf8]{inputenc}                            % Font definition and input type
\usepackage[T1]{fontenc}                                 % Font output type
\usepackage{mmap}                                        % Allows glyphs (e.g. ff) to copy properly (ASCII)
\usepackage{textcomp}                                    % Supports many symbols such as copyright
\usepackage{color}                                       % Allow for colored text, etc.
\usepackage[]{hyperref}                                  % Allow hyperlinks (internal and external)
\hypersetup{                                             % Custom hyperlink settings
    pdffitwindow=true,                                  % Window fit to page when opened
    pdfstartview={XYZ null null 1.00},                   % Fits the zoom of the page to 100%
    pdfnewwindow=true,                                   % Links in new window
    colorlinks=true,                                     % false: boxed links; true: colored links
    linkcolor=black,                                     % Color of internal links (black is necessary for printing quality)
    citecolor=black,                                     % Color of links to bibliography
    urlcolor=[rgb]{0.31,0.16,0.52},                                      % Color of external links
    pdfauthor = {Kensuke Maeba},
    pdfkeywords = {Economics, Northwestern},
    pdftitle = {CV, Kensuke Maeba},
    pdfsubject = {Curriculum Vitae},
    pdfpagemode = UseNone}

\oddsidemargin -.5in
\evensidemargin -.5in
\textwidth=6.0in
\itemsep=0in
\parsep=0in
% if using pdflatex:
\setlength{\pdfpagewidth}{\paperwidth}
\setlength{\pdfpageheight}{\paperheight}

	\addtolength{\topmargin}{-.3in}
	\addtolength{\textheight}{0.6in}

\newenvironment{list1}{
  \begin{list}{\ding{113}}{%
      \setlength{\itemsep}{.025in}
      \setlength{\parsep}{0in} \setlength{\parskip}{0in}
      \setlength{\topsep}{0in} \setlength{\partopsep}{0in}
      \setlength{\leftmargin}{0.17in}}}{\end{list}}
\newenvironment{list2}{
  \begin{list}{$\bullet$}{%
      \setlength{\itemsep}{0in}
      \setlength{\parsep}{0in} \setlength{\parskip}{0in}
      \setlength{\topsep}{0in} \setlength{\partopsep}{0in}
      \setlength{\leftmargin}{0.2in}}}{\end{list}}

\usepackage[nodayofweek]{datetime}                                    % Custom date format for date field
\newdateformat{mydate}{\monthname[\THEMONTH] \THEYEAR}   % Defining month year date format

\usepackage{fancyhdr}                                    % Used for custom page headers
\pagestyle{fancy}
\fancyhf{}
\renewcommand{\headrulewidth}{0.5pt}
\rhead{\footnotesize Kensuke Maeba \thepage} %header at the right
\headsep = 0.5cm
%% FIRST PAGE ONLY (redefine the plain pagestyle)
%\fancypagestyle{plain}{
%\fancyhf{}
%\renewcommand{\headrulewidth}{0pt}
%\headsep = 0.0cm
%%\rhead{\footnotesize Last updated \today}
%\rhead{\footnotesize \includegraphics[scale=.11]{Shorthand-Vertical-Purple}}
%}


\begin{document}
\pretolerance=10000

~\\
~\\
\name{  {\LARGE  Kensuke Maeba } \vspace*{.1in}}

\begin{resume}
\thispagestyle{plain} % to use first page footer

\begin{table}[h]
\footnotesize
\begin{tabular}{@{}p{0.20in}p{1.5in}p{1.4in}p{.7in}p{1.5in}}
& Placement Director:	& Professor Alessandro Pavan & 847-491-8266  & \href{mailto: alepavan@northwestern.edu}{alepavan@northwestern.edu}\\
& Placement Administrator:	& Lola May Ittner & 847-491-8200  & \href{mailto: econjobmarket@northwestern.edu}{econjobmarket@northwestern.edu}\\
 \end{tabular}
\end{table}

\section{Contact Information}
\vspace{.05in}
\begin{tabular}{@{}p{0.20in}p{2.75in}p{2.75in}}
 & Department of Economics           & 312-874-9284 \\
 & Northwestern University   &  \href{mailto: kensukemaeba2022@u.northwestern.edu}{kensukemaeba2022@u.northwestern.edu} \\
 & 2211 Campus Drive &  \href{http://sites.northwestern.edu/kml0807}{www.sites.northwestern.edu/kml0807}\\
 & Evanston, IL 60208  & Citizenship: Japanese
\end{tabular}

\section{Research Fields}
\begin{list1}
\item[] Primary: Development Economics
\item[] Secondary: Industrial Organization, Economics of Education
\vspace*{.05in}
\end{list1}

\section{ Education}\begin{tiny}

\end{tiny}
\begin{list1}
\item[] Ph.D., Economics, Northwestern University, 2023 (Expected)
	\begin{list2}
		\item[] Dissertation: Essays on education policies in developing countries
		\item[] Committee: Seema Jayachandran (Co-Chair), Lori Beaman (Co-Chair), Christopher Udry, Vivek Bhattacharya
	\end{list2}
\item[] M.A., Economics, Northwestern University, 2018
\item[] M.A., Economics, The University of Tokyo, 2017
\item[] B.A., Economics, Keio University, 2015
\end{list1}

\section{Fellowships \& Awards}
\begin{list1}
\item[] 2022-2023: Graduate Dissertation Fellowship
\item[] 2021-2022: Distinguished Teaching Assistant Award
\item[] 2017-2020: WCAS External Fellowship
\item[] 2017-2020: Japan Student Service Organization (JASSO) Student Exchange Support Program (Graduate Scholarship for Degree Seeking Students)
\end{list1}

\section{Teaching Experience}
\begin{list1}
\item[] Teaching Assistant, Northwestern University
	\begin{list2}
		\item[] 2022 Winter: Introduction to Microeconomics (Prof. Scott Ogawa)
		\item[] 2021 Fall: Microeconomics (Prof. Scott Ogawa)
		\item[] 2021 Winter: Introduction to Microeconomics (Prof. Sara Hernández-Saborit)
		\item[] 2020 Fall: Introduction to Applied Econometrics (Prof. Daley Kutzman)
	\end{list2}
\item[] Teaching Assistant, The University of Tokyo
	\begin{list2}
		\item[] 2017 S1: Case Study (Poverty and Development I) (Prof. Yoshito Takasaki)
		\item[] 2016 S2: Development Economics: Microeconomic Approach (Prof. Yoshito Takasaki)
	\end{list2}
\end{list1}

\section{Research Experience}
\begin{list1}
\item[] Research Assistant, Northwestern University
	\begin{list2}
		\item[] 2021 Spring: Prof. Frank Limbrock
	\end{list2}
\item[] Research Assistant, The University of Tokyo
	\begin{list2}
	        \item[] 2017-2018: Prof. Malek Mohammad Abdul
		\item[] 2016-2017: Prof. Hiroyuki Nakata
		\item[] 2016-2017: Prof. Hideo Owan
		\item[] 2016-2017: Prof. Yasuyuki Sawada
	\end{list2}
\end{list1}

\section{Conferences}
\begin{list1}
\item[] 2017: GRIPS-UTokyo Empirical Workshop, International Association for Applied Econometrics Conferences
\end{list1}

\pagebreak

\section{Job Market Paper}
\begin{list1}
\item[] ``Extrapolation of Treatment Effect Estimates Across Contexts and Policies: An Application to Cash Transfer Experiments''
\begin{list2}
\item[] Predicting the effects of a new policy often relies on existing evidence of the same policy in other contexts. However, when those contexts are not comparable, one might make predictions based on adjacent policies in the same contexts. This paper compares the performance of these approaches in the case of conditional cash transfers (CCTs). Using cash transfer programs in Malawi and Morocco, I predict the average treatment effects of Moroccan CCTs on enrollment rates based on either Malawi CCTs or Moroccan labeled cash transfers (LCTs). I show that predictions based on the Moroccan LCTs (across policies) are more accurate than the Malawi CCTs (across contexts). To shed light on what causes the difference, I estimate a dynamic model of schooling decisions under these interventions and compare the estimated parameters. I find that the returns to schooling relative to outside options drive the differential performance, even after controlling for observables in which target populations are different across the experiments. I suggest that varying outside options across the contexts shape the returns differentially.
	\end{list2}
\end{list1}


\section{Working Papers}
\begin{list1}
\item[] ``Is Workfare a Good Anti-Poverty Policy? An Assessment Based on Household Welfare, School Enrollment, and Program Expenditures''
\begin{list2}
\item[] A workfare program is a common anti-poverty policy in developing countries, which hires the unemployed for public construction work. While other anti-poverty policies such as cash transfer programs select participants based on a proxy-mean test, participants self-select into a workfare program, which could improve the efficiency of the program and participants' welfare by reducing inclusion and exclusion errors. On the other hand, a workfare program induces school dropout among the participants' children, which would potentially increase poverty rates in the future. This paper evaluates a large workfare program in India against hypothetical cash transfer programs by quantifying these aspects. Our empirical results show that under the equivalent program expenditures, the workfare program increases household welfare and enrollment rates less than the cash transfer programs. We further show that to achieve the levels of household welfare under the cash transfer programs, the workfare program needs to yield unreasonably high rates of social returns, suggesting that the cash transfer programs are preferred in terms of our evaluation metrics.
\end{list2}
\end{list1}

\section{Work in Progress}
\begin{list1}
\item[] ``How the Political Power of Teacher Unions Affects Education'' (joint with Eduardo Campillo Betancourt)
\begin{list2}
\item[] This project studies the clientelism between corporatist teacher unions and their members, and this relationship’s impact on education outcomes. Focusing on the biggest teacher union in Mexico around the 2006 presidential election, we compare schools in municipalities before and after the election, and use cross-sectional variation in the degree of the union’s support for the winning party. Our difference-in-differences estimates show an increase in the number of public school teachers both incorporated into and promoted within a pay-for-performance program, a known patronage tool that the union uses to reward teachers for their grass-roots political campaign. We also show evidence of lower test scores on national standardized exams among public schools in more affected municipalities after the election, consistent with the fact that only public school teachers were eligible for the program. These results present evidence of the deleterious consequences of large, politically-motivated unions in the Mexican context. \\
\end{list2}
\item[] ``Net Influences from Polling Officers''
\begin{list2}
\item[] In developing countries, election officers can influence voting outcomes, which deteriorates the value of democracy. However, little is known about to what extent they change voting outcomes. This paper proposes a way to quantify such an influence using voting outcomes in the 2019 India's national parliamentary election and administrative data on polling station officers. I construct a discrete choice model of voting, where a voter chooses a candidate based on his characteristics, including the influence of polling officers as an unobserved trait. The identification of the influence exploits a random assignment of the officers to polling stations. I argue that conditional on voters' and candidates' characteristics, the residual variation in vote shares at the polling station level captures the influence of the polling officers. I show the robustness of this measure by controlling for a local political environment and granular location fixed effects.  
\end{list2}
\end{list1}

\section{Other Papers}
\begin{list1}
\item[] ``Birth Order Effect Under a Cash Transfer Program'', Available at SSRN, July 2019, pre-Ph.D. thesis
\end{list1}

\section{Professional Services}
\begin{list1}
\item[] 2020: A committee member \& a discussant for EMCON
\item[] 2019: A committee member for Empirics and Methods in Economics Conference (EMCON)
\end{list1}

\section{Languages}
\begin{list1}
\item[] English (fluent), Japanese (native)
\end{list1}

\section{ References}
\vspace{.05in}
%Pick from:
\begin{tabular}{@{}p{0.20in}p{2.75in}p{2.75in}}
 & Professor Seema Jayachandran (Co-Chair)  & Professor Lori Beaman (Co-Chair) \\
 & Department of Economics \& & Department of Economics \\
 & School of Public and Intl. Affairs & Northwestern University \\
 & Princeton University   &  2211 Campus Drive \\
 & 126 Julis Romo Rabinowitz Building  &  Evanston, IL 60208 \\
 & Princeton, NJ 08544  &  \href{mailto: l-beaman@northwestern.edu}{l-beaman@northwestern.edu} \\
 &  \href{mailto:  jayachandran@princeton.edu}{jayachandran@princeton.edu} & \\
\end{tabular}
\vspace{.1in} ~\\
\begin{tabular}{@{}p{0.20in}p{2.75in}p{2.75in}}
 & Professor Christopher Udry  & Professor Vivek Bhattacharya \\
 & Department of Economics   & Department of Economics \\
 & Northwestern University   & Northwestern University \\
 & 2211 Campus Drive  & 2211 Campus Drive  \\
 & Evanston, IL 60208  & Evanston, IL 60208\\
 &  \href{mailto: christopher.udry@northwestern.edu}{christopher.udry@northwestern.edu} &  \href{mailto: vivek.bhattacharya@northwestern.edu}{vivek.bhattacharya@northwestern.edu} \\
\end{tabular}

\end{resume}
\end{document}
